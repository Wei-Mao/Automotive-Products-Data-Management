\section*{小结}
\addcontentsline{toc}{section}{小结}
上完沈教授的课后,自己就开始在网上查找关于用 PHP制作网站的视频。在开始制作网站之前,我每周边看视频边学着做,大概花了一个多月的时间, 基本上能够熟练地运用所需要的软件和需要的语言。在完成软件技术基础
这门课的大作业的过程中,我主要运用了Adobe Dreamweaver CS6、PhpStudy for IIS 2014和MySQL Workbench 6.3 CE等软件。
在完成网站大作业的过程中自己遇到了很多困难,比如:数据库的连接,数据库的各项操作,BOM 表的动态生成等等。 在编程序的过程中出现了很多错误,特别是逻辑错误, 有时查找很久才能找出来。 通过自己耐心的调试, 基本上把这些问题都解决了。在动态网页的开发中过程中,我主要以11级黄江涛学长的功能为目标,用PHP语言、JQuery等实现这些功能,此外我还引入了用户认证功能,对用户的访问做了适当的限制,提高了数据的安全性。

总的来说, 完成这次作业对自己的意义重大。这次大作业让我对网页开发流程有了一个大概的了解,首先需要制定一个学习计划,该学习学习什么软件,学习什么语言,各项功能实现的进度安排。在这次开发过程中也让我深刻地体会到了,在有限时间内,精准学习的重要性。比如,本网页的BOM表实现只需要学习JQuery基础语法、事件和遍历等功能,并不需要对JavaScript和JQuery做系统全面的学习。动态网页开发博大精深,不可能在几个月的时间内,全部学会,根据开发需求,精准学习是提高开发效率的重要途径。

由于自己能力有限, 本系统还存在一定的缺陷, 恳请沈教授不吝赐教,使本系统日臻完善,谢谢!